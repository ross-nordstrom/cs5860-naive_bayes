\section{Results}
\label{section:results}
In this section results from evaluating the Naive Bayes Classifier against various datasets and using a few different
tokenizing approaches are described.

\subsection{Datasets}
\label{subsection:datasets}
In addition to the required Dickens and Hardy books required for the assignment, some additional datasets
were taken from the UCI Machine Learning Repository \cite{uci}. Specifically, classification datasets stored in an
easy-to-use text format were selected. The datasets used are described in \ref{table:datasets}. For each dataset, I
specify which "type" it is, indicating the structure in which the data is stored. The dataset types are described in
\ref{table:datasetTypes}.


\begin{table}
    \begin{tabular}{lll}
        \hline
        \textbf{Dataset} & \textbf{Source} & \textbf{Type} \\ [0.5ex]
        \hline\hline
        SMS & UCI - SMS Spam Collection & inline \\
        Badges & UCI - Badges & inline \\
        Main & Gutenberg - Dickens \& Hardy & gutenberg \\
        \hline
    \end{tabular}
    \caption{Datasets used for this paper}
    \label{table:datasets}
\end{table}

\begin{table}
    \begin{tabular}{lll}
        \hline
        \textbf{Type} & \textbf{Description} \\ [0.5ex]
        \hline\hline
        inline & Dataset is stored as a single file in which each line represents a training point. The first word in each line is the class/category, while the rest of the line is a list of words used as the training "text blob." \\
        gutenberg & Dataset is stored as a list of directories representing classes/categories (e.g. "dickens", "hardy"). Each file within the class directories represent a training point. These files are actually books, but are abstractly considered to be "text blobs," just like the **inline** dataset type. \\
        \hline
    \end{tabular}
    \caption{Types of datasets used}
    \label{table:datasetTypes}
\end{table}



\subsection{Extending the Classifier}
\label{subsection:advancedResults}
TODO


